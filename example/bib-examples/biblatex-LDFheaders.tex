
% ---- Bibliography settings and experiments with BibLaTeX


% -- "In <proceedings>", not "In: <proceedings>" --
\renewcommand{\intitlepunct}{\addspace}
% -- And no "In" for journals. --
\renewbibmacro*{in:}{%
  \ifentrytype{article}{}{\printtext{\bibstring{in}\intitlepunct}}}

% -- Try to disallow entries from breaking across pages --
% (From http://tex.stackexchange.com/questions/43260/biblatex-no-pagebreak-in-bibliography-entry)
\patchcmd{\bibsetup}{\interlinepenalty=5000}{\interlinepenalty=10000}{}{}
% Consider also, from http://tex.stackexchange.com/questions/91686/preventing-page-breaks-from-occurring-in-bibliography-items:
% (But these are already the default settings for biblatex.)
%\patchcmd{\thebibliography}{\clubpenalty4000}{\clubpenalty10000}{}{}
%\patchcmd{\thebibliography}{\widowpenalty4000}{\clubpenalty10000}{}{}

% -- Only print editors' names if there's no author --
\AtEveryBibitem{%
 \ifnameundef{author}{}{\clearname{editor}}%
}


% -- Don't print DOI + URL --
% Specifically, print no more than one of doi, eprint, or url, whichever comes first.
% (Subject to respecting the global flags about printing these.)
% Overrides code in standard.bbx
\renewbibmacro*{doi+eprint+url}{%
  \ifboolexpr{
    test {\iftoggle{bbx:doi}}
    and
    not test {\iffieldundef{doi}}
  }
    {\printfield{doi}}
    {    
      \ifboolexpr{
        test {\iftoggle{bbx:eprint}}
        and
        not test {\iffieldundef{eprint}}
      }
      {\usebibmacro{eprint}}
      {
          \ifboolexpr{
            test {\iftoggle{bbx:url}}
            and
            not test {\iffieldundef{url}}
      	}
	{\usebibmacro{url+urldate}}
        {}      
      }
    }
}


% -- Name order: can choose among last-first, first-last, and last-first/first-last (which treats 1st name differently) --
%\DeclareNameAlias{default}{last-first/first-last}
% (Decided to keep it as before: first-last but with first names initialized.)


% --- Titles ---

% -- Remove quotes from titles for inproceedings and some others. --
\DeclareFieldFormat
  [article,inbook,incollection,inproceedings,patent,thesis,unpublished]
  {title}{#1\isdot}


% -- Put conference paper titles into sentence case (only first word is capitalized) --

% Note: because the following code uses the symbol '@', must surround it with \makeatletter / \makeatother to prevent parse errors
\makeatletter

% Define "sentencecase" using \MakeSentenceCase
% (borrowed from IEEE's style file: http://mirrors.ctan.org/macros/latex/contrib/biblatex-contrib/biblatex-ieee/ieee.bbx). 
% Affects \printfield (only), called as \printfield[sentencecase]{some-particular-fieldname}.
\DeclareFieldFormat[article,inbook,incollection,inproceedings,patent,thesis,unpublished]{sentencecase}{\bbx@colon@search{\MakeSentenceCase*}{#1}}

% For sentencecase to work, also need \bbx@colon@search from that file:
% [I believe this function searches for colons, so that we can capitalize the next word]
\newcommand*{\bbx@colon@search}[2]{%
  \bbx@colon@search@auxi\empty#2: \stop{#1}%
}
\long\def\bbx@colon@search@auxi#1: #2\stop#3{%
  \expandafter\bbx@colon@search@auxii\expandafter{#1}{#3}%
  \ifblank{#2}
    {}
    {%
      : %
      \bbx@colon@search@auxi\empty#2\stop{#3}%
    }%
}
\newcommand\bbx@colon@search@auxii[2]{#2{#1}}


\makeatother

% Now that "sentencecase" is defined, need to have it be called whenever a title is printed (from one of the entry types specified in \DeclareFieldFormat).
\renewbibmacro*{title}{%
  \ifboolexpr{
    test {\iffieldundef{title}}
    and
    test {\iffieldundef{subtitle}}
  }
    {}
    {\printtext[title]{%
       \printfield[sentencecase]{title}%
       \setunit{\subtitlepunct}%
       \printfield[sentencecase]{subtitle}}%
     \newunit}%
  \printfield{titleaddon}}


% Next, want to have booktitles (proceedings names) use actual title case.

% -- Put all other titles into a real title case --

% Note: biblatex.def by default prints most titles in format "titlecase", but defines "titlecase" to mean "do nothing." 
% So, one option is to re-define titlecase:
%\DeclareFieldFormat{titlecase}{\bbx@colon@search\MakeCapital{#1}}  % no change, because it just capitalizes the very 1st letter
%\DeclareFieldFormat{titlecase}{\bbx@colon@search\MakeUppercase{#1}}    % test something extreme. Yup, that shows up.

\usepackage{titlecaps}
\Addlcwords{for a is but and with of in as the etc on to if from}
\DeclareFieldFormat{titlecase}{\titlecap{#1}}			% eureka! It works! Beautiful!






% -- Change formatting of "volume, issue" for journal articles to follow ACM's style: "\italics{journal name, volume} (issue)"
% Actually, I'm instead doing: "\italics{journal name, volume}(issue)". I might want to change it further to avoid an awkwardly parenthesized year right afterwards.
% This function affects articles. Technically I should also change \newbibmacro*{title+issuetitle}, used for periodicals, but I don't have any in my bib file.
\renewbibmacro*{volume+number+eid}{%
  \setunit{\addcomma\addspace}%				% See documentation section 4.11.7.1 as well as the definition of \setunit to further fix this.
  \printfield[emph]{volume}%
  \printfield[parens]{number}%
  \setunit{\addcomma\addspace}%
  \printfield{eid}}

% Note: if the journal ends in a period (as an abbreviation), neither this nor the original adds a comma after it. Maybe \addcomma is too polite of a command?

% Note: the way the journal gets italicized is: bibmacro{journal} calls \printfield{journaltitle}, which goes through 
% \DeclareFieldFormat{journaltitle}{\mkbibemph{#1}}

% Note: here's why "volume" isn't italicized by default. 
%\DeclareFieldFormat{volume}{\bibstring{volume}~#1}% volume of a book
%\DeclareFieldFormat[article,periodical]{volume}{#1}% volume of a journal
% We can change that:
%	\DeclareFieldFormat[article,periodical]{volume}{\mkbibemph{#1}}% volume of a journal
% And put "number" in parens
%	\DeclareFieldFormat[article,periodical]{number}{\mkbibparens{#1}}% number of a journal
% But I might as well save code by putting these changes in the function above.

% -- Omit the string "vol." from inproceedings --
% By default, already omitted from articles & periodicals
\DeclareFieldFormat[inproceedings]{volume}{#1}% volume of a journal


% -- Can I remove "USA" from locations / venues? --
% Ah, no, because that only works when using biber. (Could try it later.)
%\DeclareSourcemap{ 
%    \maps[datatype=bibtex]{
% 	\map{
%		\step[fieldsource=location,
%			match=\regexp{(.+),\s+USA\z},
%			replace=\regexp{$1}]
%		\step[fieldsource=venue,
%			match=\regexp{(.+),\s+USA\z},
%			replace=\regexp{$1}]
%	}
%    }
%}


% -- Change the formatting of conference proceedings to something more sensible --
% Try for this: "<authors>. <title>. In <eventtitleaddon>: <booktitle> (<conference location, date>), 
% <publisher> <series> <volume> (<number>), pp. <pages>. <doi or url>."
% But only print the "publisher" section if there's a series.

% Original code was in standard.bbx
\DeclareBibliographyDriver{inproceedings}{%
  \usebibmacro{bibindex}%
  \usebibmacro{begentry}%
  \usebibmacro{author/translator+others}%
  \setunit{\labelnamepunct}\newblock
  \usebibmacro{title}%
  \newunit
  \printlist{language}%
  \newunit\newblock
  \usebibmacro{byauthor}%
  \newunit\newblock
  \usebibmacro{in:}%
% old
%  \usebibmacro{maintitle+booktitle}%
%  \newunit\newblock
%  \usebibmacro{event+venue+date}%
%  \newunit\newblock
%  \usebibmacro{byeditor+others}%
%  \newunit\newblock
%  \iffieldundef{maintitle}
%    {\printfield{volume}%
%     \printfield{part}}
%    {}%
%  \newunit
%  \printfield{volumes}%
%  \newunit\newblock
%  \usebibmacro{series+number}%
%  \newunit\newblock
%  \printfield{note}%
%  \newunit\newblock
%  \printlist{organization}%
%  \newunit
%  \usebibmacro{publisher+location+date}%
%  \newunit\newblock
%  \usebibmacro{chapter+pages}%
% 
% new
  \usebibmacro{maintitle+confseries+booktitle}%  	% "series: booktitle". with macros redefined to include the series
  \usebibmacro{event+venue+date}%   	  		% "(location, year)". but redefining it to actually print date, not eventdate
%  \newunit								% note: standard unit punct for if some other field comes next...
  \newunit\newblock
  \usebibmacro{byeditor+others}%
  \newunit\newblock
  \printfield{note}%
  \newunit\newblock
  \printlist{organization}%		% (usually empty)
% end "usually empty" fields
    \setunit{\addcomma\addspace}%			% ...but in the usual case, just ", ". It's what always precedes publisher.
  \usebibmacro{pub+ser+vol+num}%			% includes "<publisher> <series> <volume> (<number>), pp. <pages>"
   \setunit{\addspace}
  \usebibmacro{chapter+pages}%
  \newunit\newblock
  \printfield{note}%			% (usually empty)
% end changes
  \newunit\newblock
  \iftoggle{bbx:isbn}
    {\printfield{isbn}}
    {}%
  \newunit\newblock
  \usebibmacro{doi+eprint+url}%
  \newunit\newblock
  \usebibmacro{addendum+pubstate}%
  \setunit{\bibpagerefpunct}\newblock
  \usebibmacro{pageref}%
  \newunit\newblock
  \iftoggle{bbx:related}
    {\usebibmacro{related:init}%
     \usebibmacro{related}}
    {}%
  \usebibmacro{finentry}}

% new:  "<publisher> <series> <volume> (<number>)". But, skip all of it unless there's a series.
\newbibmacro*{pub+ser+vol+num}{%
\iffieldundef{series}
    {}
  {\usebibmacro{publisher}%    				% newly defined below, includes only publisher
  \setunit*{\addspace}					% using \setunit* for the next few lines so that the above ", " is still what's used when the next thing prints
  \printfield{series}%
  \setunit*{\addspace}
  \printfield{volume}%
  % \setunit*{\addspace}
  \printfield[parens]{number}}%
}

% For inproceedings entries, a new macro that makes booktitle include the series.
\newbibmacro*{maintitle+confseries+booktitle}{%
  \iffieldundef{maintitle}
    {}
    {\usebibmacro{maintitle}%
     \newunit\newblock
     \iffieldundef{volume}
       {}
       {\printfield{volume}%
        \printfield{part}%
        \setunit{\addcolon\addspace}}}%
  \usebibmacro{confseries+booktitle}}%

% Re-modified to use eventtitleaddon as the field where the conference acronym is stored.
\newbibmacro*{confseries+booktitle}{%
  \ifboolexpr{
    test {\iffieldundef{booktitle}}
    and
    test {\iffieldundef{booksubtitle}}
    and
    test {\iffieldundef{eventtitleaddon}}
  }
    {}
    {\printtext[booktitle]{%
	\iffieldundef{eventtitleaddon}
              {}
              {\printfield{eventtitleaddon}% 
              \setunit{\addcolon\addspace}}%
           \printfield[titlecase]{booktitle}%
           \setunit{\subtitlepunct}%
           \printfield[titlecase]{booksubtitle}}%
     }%
  \setunit{\addcomma\addspace}%
  \iffieldundef{booktitleaddon}
  {}% 	% for debugging: \printtext{(hello)} }
  {%\setunit{\addcomma\addspace}%
  \printfield{booktitleaddon}%
  \newunit}
  }

  
% Make venue + date mean the publication date. Also, no longer actually printing any "event" fields here.
\renewbibmacro*{event+venue+date}{%
  \ifboolexpr{  % put everything in parens, so check if there's anything to put inside
    test {\iffieldundef{venue}}
    and
    test {\iffieldundef{year}}
%    and
%    test {\iffieldundef{eventtitle}}
%    and 
%    test {\iffieldundef{eventtitleaddon}}
  }
    {}
    {\setunit{\addspace}%
     \printtext[parens]{%
%  \printfield{eventtitle}%
%  \newunit
%  \printfield{eventtitleaddon}%
       \printfield{venue}%
       \setunit*{\addcomma\addspace}%
       \usebibmacro{date}}}%
  \newunit}


% -- For inproceedings, omit publisher's location and date --
\newbibmacro*{publisher}{%
  \printlist{publisher}%
  \setunit*{\addcomma\addspace}%
  \newunit}



% --- End bibliography settings
